\documentclass[12pt,a4paper,draft]{article}

\usepackage[romanian]{babel}
\usepackage[T1]{fontenc}
\usepackage[utf8x]{inputenc}
\usepackage{graphicx}

\author{Cristian Paraschiv}
\title{Cross-site Scripting}

\bibliographystyle{plain}

\begin{document}
	\section{Introducere}
	Răspândirea internetului către o plajă din ce în ce mai largă de audien\-ță a creat o nouă putere financiară: cea a tranzacțiilor pe Internet. Odată cu această înmulțire financiară a crescut și atenția acordată de persoane răuvoitoare acestor tranzacții. Ținând cont de aceste circumstanțe domeniul securității informaționale online s-a dezvoltat continuu încercând să țină pasul cu creativitatea hacker-ilor. Datorită faptului că modalitatea dominantă de efectuare a tranzacțiilor online este prin intermediul unui browser, acesta a devenit ținta principală a atacurilor hacker-ilor.
	
	Browser-ul reprezintă un program care interpretează informațiile și in\-strucțiunile primite după o cerere făcută în prealabil către un server. În prezent orice dispozitiv ce poate fi conectat la internet are sub o formă sau alta un browser, fie că face parte dintr-o aplicație ce îndeplinește mai multe scopuri, fie că este un program de sine-stătător.
	
	În momentul de față sunt disponibile o varietate destul de mare de \linebreak browser-e, fiecare cu mai multe versiuni, fapt ce creează multe probleme de securitate celor care folosesc versiuni neactualizate ale acestora. De asemenea sunt exploatate problemele de securitate create din lipsa experienței  programatorilor de aplicații web.
	
	\section{Vulnerabilități}	
	
\end{document}